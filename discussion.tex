% !TEX root = ut-thesis.tex

\chapter{Discussion}

\section{Limitations}

Cystic fibrosis

  - Existing approaches remain limited in their ability to predict disease-causing variants. In a study of 1571 mutations of the CFTR gene causing cystic fibrosis, (SIFT, PolyPhen, PANTHER) \cite{dorfman_common_2010}


Long QT syndrome

  - Assessment of the predictive accuracy of five in silico prediction tools, alone or in combination, and two metaservers to classify long QT syndrome gene mutations.

  - http://www.ncbi.nlm.nih.gov/pubmed/25967940



\section{Extensions}

- Predict homo-oligomers, since this is the predominant form of oligomerization in proteins.

- Multiple amino acid substitutions + insertions / deletions

- Alternative splicing / aberrant splicing



\section{Applications}

 - Predicting deleterious mutations

 - Stabilizing proteins

 - drugging protein-protein interfaces \cite{wells_reaching_2007}


\section{Domain definitions}

eSCOP

Gene3D

\section{Homology modelling}

There are several approaches we could take to improve the quality of the homology models:

\begin{itemize}
\item Use sequence profiles (e.g. Pfam or Gene3D) to guide the alignment.
\item Use multiple templates when building the homology models.
\item Create multiple models and choose the one with the highest DOPE score.
\item Refine the model using molecular dynamics.
\end{itemize}

Long-term MD is not useful for optimizing structures in most cases \cite{raval_refinement_2012}.



\section{Sequence features}

Improvement to sequence features

\begin{itemize}
\item Use covariation between amino acids in addition tho the conservation score to predict the impact of mutations, as described by Kowarsch et. al. \cite{kowarsch_correlated_2010}.
\end{itemize}



\subsection{Protein-DNA/RNA interactions}

ProNIT



\subsection{Protein-ligand interactions}

Platinum: Protein-ligand affinity change upon mutation database.

  - http://bleoberis.bioc.cam.ac.uk/platinum/


BioLiP is a semi-manually curated database for high-quality, biologically relevant ligand-protein binding interactions.

  - http://zhanglab.ccmb.med.umich.edu/BioLiP/

  - The structure data are collected primarily from the Protein Data Bank, with biological insights mined from literature and other specific databases.

\subsection{Predicting PPI}

PRISM: Protein interaction by structure matching

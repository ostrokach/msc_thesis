% !TEX root = msc_thesis.tex

\chapter{Discussion} \label{ch:discussion}

The primary objective of this project was to extend ELASPIC and make it possible to predict the effect of mutations on protein stability and protein-protein interaction affinity on a genome-wide scale. We were able to meet this objective with a reasonable amount of success. We implemented ELASPIC as an easy to install and fully automated pipeline, which can scale from a single mutation in a user-provided PDB file to hundreds of thousands of mutation affecting most proteins in the genome. We calculated $\Delta \Delta G$ values for almost one million mutations implicated in human diseases or found in cancers, which, to our knowledge, makes this the first study to evaluate the structural impact of mutations at such a large scale. It is now possible to use ELASPIC as part of a toolbox for annotating variants discovered through high-throughput sequencing, since evaluating a mutation takes under several minutes if the Provean supporting set and homology models have been precalculated. With the work of Daniel Witvliet \textit{et al.} \cite{witvliet_elaspic_2016}, it is also possible to use ELASPIC through a webserver.

% I think the main To-Do is to update the discussion quite significantly. You have to discuss in a bit more detail the differences in performance of elaspic vs provean (or other sequence-based predictors) and talk a tiny bit about biochemical vs phenotypic prediction.

The secondary objective of this project was to use structural information provided by ELASPIC to make accurate and informative predictions regarding the phenotypic effect of mutations. This objective was met with much less success. Provean remains more accurate than ELASPIC in predicting whether a mutation is associated with a disease, as seen in the performance of the two tools on the validation and test subsets of the Humsavar, ClinVar and COSMIC datasets (Figures \ref{fig:core_validation} and \ref{fig:interface_validation}). ELASPIC does achieves better performance than Provean in differentiating between loss-of-function and gain-of-function mutations (Figures \ref{fig:validation_cancer_full} and \ref{fig:validation_cancer_high_confidence}). However, the difference in performance is not substantial, and it does not improve our ability to distinguish between tumour suppressor genes and oncogenes (Figures \ref{fig:validation_cancer_bygene_full} and \ref{fig:validation_cancer_bygene_high_confidence}). The sequence conservation score already take into account most deleterious effects of mutations, including changes in protein stability and protein-protein interaction affinity, as well as other effects that not considered by ELASPIC, such as the disruption of active sites, ligand binding pockets and residues involved in cooperative interactions that regulate the activity of the protein. This is consistent with a recently published study, where the authors trained a machine learning algorithm called Variant Interpretation and Prediction Using Rosetta (VIPUR) to predict the deleteriousness of mutations using the Provean score, $\Delta \Delta G$ predicted by Rosetta and other structural and sequential features \cite{baugh_robust_2016}. VIPUR achieves an area under the receiver operating characteristic (ROC) curve of 0.831, while Provean achieves an area under the ROC curve of 0.819 on the same dataset. Overall, it seems that the additional information provided by tools such as ELASPIC and VIPUR does not justify the extra complexity and computational costs required to evaluate the structural impact of mutations.

Several observations made during the training and validation of ELASPIC predictors warrant further discussion. We found that mutations falling inside protein-protein interfaces are more likely to be involved in disease, but the effect size that we observe is much smaller than what was reported previously. In the case of the UniProt humsavar database, we observe that 8.4\% of deleterious mutations and 3.5\% of polymorphic mutations fall inside protein-protein interfaces. If we restrict polymorphic mutations to the set of proteins also affected by deleterious mutations, the value increases to 4.3\%, likely because proteins implicated in disease are better studied, have more structural templates and, consequently, have more homology models of protein-protein interaction. In contrast, a previous report indicates that 57\% of disease mutations and 8\% of polymorphic mutations disrupt protein-protein interactions \cite{sahni_widespread_2015}. It is likely that we underestimate the fraction of disease and polymorphic mutations that affect protein-protein interaction, since our structural coverage of the protein-protein interaction network is very incomplete. However, the discrepancy between the 2-fold over-representation of interface mutations in disease that we observe in our work, an the 7-fold over-representation of interface mutations in disease that are reported by Sahni \textit{et al.}, should be examined further. One possible reason for the discrepancy is a bias in the experiments performed by Sahni \textit{et al.}. For example, the HI-II-14 human interactome map used in yeast two-hybrid experiments may be enriched in protein-protein interactions that are involved in disease, making it appear that disease mutations break more protein-protein interactions than polymorphisms. Also, Sahni \textit{et al.} report that disease mutation tend to alter the protein-chaperone interaction profile much more than polymorphic mutations, and increased interaction with chaperones could block some of the protein interaction interfaces, preventing native protein-protein interactions from occurring. Another explanation for the discrepancy could be that we create inaccurate homology models for many of the protein-protein interactions, and therefore do not capture accurately the effect of disease mutations on those interactions.

We found that interface mutations affecting proteins involved in multiple protein-protein interactions are usually ``edgetic'', in that they affect only one of the interactions. For example, in the case of proteins in the UniProt humsavar database that are involved in multiple interactions, 64\% of deleterious mutations and 56\% of polymorphic mutations affect only one of the interactions, and less than 1\% of deleterious and polymorphic mutations affect all interactions. In the case of mutations affecting multiple interactions, we found that the average $\Delta \Delta G$ caused by a mutation is a better predictor of mutation deleteriousness than the difference in $\Delta \Delta G$ between the interface that is the most affected and the interface that is the least affected. This suggests that the ``edgetic'' nature of disease mutations is more due to the fact that proteins use distinct and non-overlapping interfaces to interacting with different partners, than due to an inherent propensity of those mutations to be selective in the interactions that they disrupt.

Results of feature elimination suggest that most features calculated by ELASPIC are either highly correlated or contain little information regarding protein stability or protein-protein interaction affinity (Figures \ref{fig:feature_elimination_core} and \ref{fig:feature_elimination_interface}). We do observe a consistent trend in that, for both core and interface predictors, the features that remain at the end of feature elimination include an electrostatic term, a van der Waals term, a solvation term and an entropic term (see Tables \ref{tab:core_features} and \ref{tab:interface_features}). This is consistent with a previous study by Benedix \textit{et al.} \cite{benedix_predicting_2009}, where the authors achieve good accuracy in predicting the effect of mutations on protein stability and protein-protein interaction affinity using only five energy terms: i) electrostatic energy between charged residues, calculated using Coulomb's law, ii) van der Waals energy between residues, calculated using the Lennard Jones equation, iii) solvation energy of polar residues, calculated by solving the Poisson-Boltzmann equation, iv) solvation energy of hydrophobic residues, calculated as a linear function of the solvent accessible surface area, and v) entropic energy, calculated using the quasiharmonic approximation proposed by Schlitter \cite{schlitter_estimation_1993}. This suggests that we could replace FoldX, which we currently use as a ``black box'' to calculate many of the ELASPIC features, with the energy terms described by Benedix \textit{et al.}, while maintaining the same performance. This would grant us much better control over the ELASPIC pipeline, since we would know exactly how each feature is calculated and would be able to analyze in detail why ELASPIC makes incorrect predictions for some of the mutations.

Benedix \textit{et al.} \cite{benedix_predicting_2009} also report that the accuracy of the predictions can be substantially improved by generating an ensemble of structures using CONCOORD, and using the average of the energies calculated for the wildtype and mutant structures to make the final prediction. They use the same dataset to fit the four parameters of their model, and to test their model, so it is difficult to compare directly the accuracy that they report to the accuracy of ELASPIC. Nevertheless, when using an ensemble of 300 structures, Benedix \textit{et al.} achieve a Pearson correlation coefficient of 0.75 for core mutations and 0.79 for interface mutations, which is likely higher than what would be achieved by ELASPIC on the same dataset (see Figures \ref{fig:crossvalidation_performance_core} and \ref{fig:crossvalidation_performance_interface}). It seems probable that the performance of ELASPIC could likewise be improved by averaging the energy terms calculated for an ensemble of protein conformations. The downside of this approach would be increased computational cost, although it should still be much faster than alchemical techniques such as thermodynamic integration, while achieving similar performance \cite{seeliger_protein_2010}.

As discussed above, using structural information to assist with the prediction of the phenotypic effects of mutations offers little gain in accuracy and incurs a significant computational cost. On the other hand, using sequential information to assist with the prediction of the structural effects of mutations offers a significant gain in accuracy while adding relatively little computational cost. Case in point, the use of the Provean score is likely the main reason why ELASPIC can consistently outperform FoldX on different training, validation and test datasets. Calculating the Provean score can be done in seconds, if the Provean supporting set is available, and in about the same amount of time that it takes to create a homology model, if the Provean supporting set is not available. The Provean score appears to contain information that is distinct and complementary to the information contained by energy terms. While a mutation that has little effect on the structure of the protein may still be deleterious, in the majority of cases, a mutation is deleterious either because it disrupts the stability of the protein or because it disrupts the stability of interactions involving the protein. In those cases, the Provean score takes into account the effect of the mutation on all functional forms of the protein, rather than on a single snapshot that is the crystal structure or the homology model. Thus, the most promising avenue for future work appears to be in adding more sequential features to further improve the accuracy of the ELASPIC predictors. The performance of ELASPIC core and interface predictors on the training sets is highly correlated with their performance on the validation sets, as was shown in Sections \ref{sec:gridsearch} and \ref{sec:feature_elimination}. Since the validation sets contain 10 to 100 times more mutations than the training sets, and are much more diverse, they could contribute much information to the ELASPIC predictors. Some possible ways in which this could expand and incorporate more information into the ELASPIC training set are discussed in Chapter \ref{ch:future_directions}.

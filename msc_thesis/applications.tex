\mychapter{3}{Applications}

\section{Homology modelling of domains and domain interactions of the human proteome}

\begin{figure}[H]
	\centering
	\includegraphics[scale=0.3]{elaspic_statistics/missing_template}
	\includegraphics[scale=0.3]{elaspic_statistics/missing_model_protein}
	\includegraphics[scale=0.3]{elaspic_statistics/structural_coverage_hist}
	\caption{\textbf{Left}: statistics of how many homology models we were able to calculate. \textbf{Right}: structural coverage for proteins with at least one domain with a structural model.}
\end{figure}



\section{Comparing the energetic effect of benign and deleterious mutations}



\section{Alanine scanning of protein-protein interfaces}

- Show a histogram of $\Delta \Delta G$ values for all interfaces.

- Predict whether a peptide is going to be anti-proliferative based on the total absolute $\Delta \Delta G$ score over the interface.



\begin{figure}[H]
	\centering
	\includegraphics[scale=0.24]{image98}
	\caption[pipeline]{The drop-off score is higher for peptides that have a higher sum of the absolute interface FoldX energy.}
\end{figure}





